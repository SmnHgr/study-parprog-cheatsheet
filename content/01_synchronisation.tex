\section{Synchronisation}

\textcolor{blue}{Mutual Exclusion} Gegenseitiger Ausschluss

\textcolor{blue}{InterruptedException} \textcolor{red}{nur bei Waiting-Mechanismen} (\lstinline{wait()}, \lstinline{acquire()}, nicht aber bei \lstinline{notify()}, \lstinline{notifyAll()}, \lstinline{release()}

\lstinline{Thread.yield()} laufender Thread gibt Prozessor frei und direkt wieder ready \textcolor{red}{gibt Monitor-Lock nicht frei} \lstinline{currentThread()} gerade ausführende Thread-Instanz \lstinline{t1.setDaemon(true)} markiert Thread als Daemon

\textbf{Monitor (Threads)}

Monitor mit einem inneren Warteraum

\textcolor{blue}{Überholproblem} keine garantierte Reihenfolge/Fairness

\lstinline{notify()} informiert/weckt nicht zwingend Thread welcher Bedingung erfüllt, für Uniform Waiters: 1 semantische Bedingung (Bedingung interessiert jeden wartenden Thread), One-In-One-Out (höchstens ein wartender Thread kann weitermachen)

\lstinline{notifyAll()} verschiedene Wartebedingungen im Monitor, mehrere Threads können möglicherweise weitermachen

\textcolor{blue}{Signal and Continue} Monitor-Lock wird nicht direkt nach notify weitergegeben! Signalisierender Thread behält Monitor und macht weiter

\textcolor{blue}{Spurious Wakeup} Thread wird aufgeweckt, aber Bedingung, auf die Thread gewartet hat, ist nicht erfüllt

\textbf{Semaphore}

hoch und runterzählbar, begrenzte Ressource, Durchlässe gezählt, blockiert $\le 0$

\begin{lstlisting}
// Wait()
Semaphore x = new Semaphore(0);
await() { x.acquire(N); x.release(N); }
toggle() { x.release(); }
\end{lstlisting}

Überlegung: Wie viele Limitationen gibt es? Je Limit. ein Semaphor

\begin{lstlisting}
class CyclicBarrier {
    private int participants; private int counter = 0;
    private Semaphore mutex = new Semaphore(1);
    private Semaphore entry = new Semaphore(0); // private int entered = 0;
    private Semaphore exit = new Semaphore(0); // private int exited = 0;

    public CyclicBarrier(int participants) { this.participants = participants; }

    public void await() throws InterruptedException {
        mutex.acquire();
        counter++; // entered++;
        // Warten bis letzter in Barriere eintritt
        if (counter == participants) { // entered == participants
            entry.release(participants); } // exited = 0; notifyAll();
        mutex.release();
        entry.acquire(); // while (entered < parties){ wait(); }
        // Warten bis alle Barriere verlassen haben (für zyklische Verwendung)
        mutex.acquire();
        counter--; // exited++;
        if (counter == 0) { // exited == participants
            exit.release(participants); } // entered = 0; notifyAll();
        mutex.release();
        exit.acquire(); } } // while (exited < participants) { wait(); }
\end{lstlisting}

\textbf{Lock \& Condition} Monitor mit mehreren Warteräumen, Fairness

\textbf{Read-Write Lock} erlaubt kein Write und parallele Aktion, immer mit finally unlock

\textbf{CountDownLatch} zeitliche Synchronisation, nicht zyklisch, warten \& signalisieren sind getrennt, kann nur runterzählen, hat Ressourcen \textcolor{red}{keine Kontrolle wie viele Threads durch gehen}, blockiert $> 0$ \lstinline{await()} warten bis Countdown = 0 \lstinline{countDown()} zähler um 1 dekrementieren \lstinline{getCount()} gibt aktuelle Anzahl an

\textbf{CyclicBarrier} \lstinline{await()} \lstinline{BrokenBarrierException} ist zyklisch, \textcolor{red}{Anzahl Teilnehmende müssen angegeben werden}

\subsection{Gefahren}

\textbf{Race Conditions} Programm verhaltet sich nicht mehr richtig (tut nicht was es soll), nicht/zu wenig atomar (zu wenig Synchronisation), Lost updates möglich (wenn Sichtbarkeit nicht garantiert, ungültige Umordnungen, falsches Programmverhalten, Lesen und Schreiben ist nicht atomar) Sichtbarkeitsproblem (wenn es entscheidend ist ob Thread 1 zuerst auf x schreibt oder Thread 2 x liest, mind. 1 Write-Access) \textcolor{red}{get-Statements} \textcolor{red}{Berechnungen in synchronized Parameter-Liste}

\textbf{Data Race} nebenläufiger unsynchronisierter R-W/W-W Zugriff auf gleiche Variable (Speicher), keine garantierte Sichtbarkeit/Ordnung $\rightarrow$ \lstinline{volatile} und alle Synchronisationprimitiven lösen auf, \textcolor{darkGreen}{Atomarität behebt Data Race}

\textbf{Deadlock} Gegenseitiges Aussperren von Threads, Ressource wird nicht mehr frei \textcolor{red}{Geschachtelte Locks} Locks in Locks \textcolor{red}{zyklische Warteabhängigkeiten} (in synchronized anderes Objekt verwendet) Thread T1 ruft a.fn(b) auf, Thread T2 ruft b.fn(a) auf, T1 sperrt a, T2 sperrt b, T1 ruft geschachtelt ''b'' auf (ist durch T2 blockiert), T2 ruft geschachtelt ''a'' auf (ist durch T1 blockiert) \textcolor{darkGreen}{C\# mit Task-Abhängigkeiten nicht möglich}, \textcolor{darkGreen}{Lineare Sperrordnung} nur geschachtelt in aufsteigender Reihenfolge sperren

\textbf{LiveLocks} Threads haben sich gegenseitig permanent blockiert, führen aber noch Warteinstruktionen aus (verbrauchen CPU während Deadlock)

\textbf{Starvation} Kontinuierliche Fortschrittsbehinderung von Threads wegen Fairness-Problemen, Bedingung wird von wartenden Thread erfüllt aber immer wieder überholt, \textcolor{red}{Java Monitor sehr anfällig}, Atomare Operationen und while auch! \textcolor{darkGreen}{Fairness einschalten}

\subsection{Bewusster Synchronisationsverzicht}

\textbf{Immutability (Unveränderlichkeit)} Objekte mit nur lesendem Zugriff \textcolor{blue}{Instanzvariablen sind alle final}

\textbf{Confinement (Einsperrung)} Objekt gehört nur einem Thread zu einer Zeit



%\subsection{Race Condition}
%Mehrere Threads lesen dasselbe Objekt und überschreiben diesen Wert basierend auf der erhaltenen Antwort (Bsp. \lstinline{deposit()} ist Critical Section: zuerst \lstinline{balance} lesen dann \lstinline{+= amount}) Zwischen Lesen und Schreiben kann ein anderer Thread die \lstinline{balance} geändert haben $\rightarrow$ Gegenseitiger Ausschluss (Mutual Exclusion)

%\begin{lstlisting}
%// A,B Ausgaben können durcheinander sein (zeitlich entkoppelt, Threads laufen beliebig verzahnt/parallel)
%public class MultiThreadTest {
%    public static void main (String[] args ) {
%        // Thread-Implementierung
%        var a = new Thread(Thread(() -> multiPrint("A"));
%        var b = new Thread(Thread(() -> multiPrint("B"));
%        a.start(); b.start(); // richtiger Thread wird hier erzeugt und gestartet (führt run()-Methode des Runnable Interface aus)
%        a.join() /* blockiert solange Thread a läuft */ b.join(); }
%    static void multiPrint (String label) {
%        for (int i = 0; i < 10; i++) {
%            System.out.println(label + ": " + i);
%            Thread.sleep(100) // versetzt laufenden Thread in Wartezustand
%        } } }
%\end{lstlisting}


%\begin{minipage}[t]{0.5\linewidth}
%    \begin{lstlisting}
%// Explizite Runnable-Implementation
%class SimpleLogic implements Runnable {
%    @Override
%    public void run () {
%        // thread behavior
%    } }
%var myThread = new Thread(new SimpleLogic());
%myThread.start();
%    \end{lstlisting}
%\end{minipage}
%\begin{minipage}[t]{0.5\linewidth}
%    \begin{lstlisting}
%// Sub-Klasse von Thread
%class SimpleThread extends Thread {
%    @Override
%    public void run () { // thread behavior } }
%var myThread = new SimpleThread(); myThread.start();
%    \end{lstlisting}
%\end{minipage}



%\subsection{Synchronisationsmechanismen/-primitiven}

%\subsubsection{Synchronized (Monitor-Lock) - Java}

%\begin{lstlisting}
%// deposit und withdraw sind im gegenseitigem Ausschluss
%class BankAccount {
%    private int balance = 0;
%    public synchronized void deposit (int amount) {
%        this.balance += amount; // kritischer Abschnitt
%        notifyAll(); } // Wecke alle im Monitor wartenden Threads
%    public synchronized boolean withdraw (int amount) {
%        while (amount > this.balance) { wait(); } // gibt Lock temporär frei und wartet auf Bedingung (im Monitor Model)
%        this.balance -= amount; }
%// synchronized(object){ statements } ist eine explizite Angabe, auf welcher Instanz gelockt wird:
%public void deposit(int amount) { // explizite Angabe auf welcher Instanz gelockt wird
%    synchronized(this) {this.balance += amount;} }
%\end{lstlisting}

%\lstinline{notify()} signalisiert/weckt nur einen beliebigen (zufälligen), wartenden Thread (Achtung: wartet vielleicht auf andere Bedingung, führt zu ewigem Warten)

%\begin{lstlisting}
%class BoundedBuffer <T> {
%  private Queue<T> queue = new LinkedList<>();
%  private int limit = 1; // or initialize in constructor
%  public synchronized void put(T item) throws InterruptedException {
%    while (queue.size () == limit) { wait(); } // await non full
%    queue.add(item); notifyAll(); } // signal non empty
%  public synchronized T get() throws InterruptedException {
%    while (queue.size () == 0) { wait(); } // await non empty
%    var item = queue.remove(); notifyAll(); // non full
%    return item; } }
%\end{lstlisting}

%\subsubsection{Semaphor - Java}
%
%\begin{lstlisting}
%// Mutex Semaphore als alternative zu synchronized
%class BoundedBuffer <T> {
%  private Queue<T> queue = new LinkedList<>();
%  private Semaphore upperLimit = new Semaphore(CAPACITY, true); // Initialisierung mit cap freien Ressourcen, true = fair
%  private Semaphore lowerLimit = new Semaphore(0, true);
%  // private Semaphore mutex = new Semaphore(1, true); // ersetzt synchronized/Monitor-Lock für Mutual exclusion
%  public void put(T item) throws InterruptedException {
%    upperLimit.acquire(); // bezieht freie Ressource (oder wartet, wenn keine verfügbar)
%    synchronized (queue) { queue.add (item); }
%    // mutex.acquire(); queue.add(item); mutex.release();
%    lowerLimit.release(); } // gibt Ressource frei, benachrichtigt Wartende
%  public T get() throws InterruptedException {
%    T item;
%    lowerLimit.acquire();
%    synchronized (queue) { item = queue.remove(); }
%    // mutex.acquire(); T item = queue.remove(); mutex.release();
%    upperLimit.release();
%    return item; } }
%\end{lstlisting}
%
%\lstinline{acquire(int permits)} wartet, solange Zähler $<$ permits ist
%
%\subsubsection{Lock \& Condition - Java}
%\begin{lstlisting}
%class BoundedBuffer<T> {
%  private Queue<T> queue = new LinkedList<>();
%  private Lock monitor = new ReentrantLock(true); // Lock-Objekt, Sperre für Eintritt in Monitor, true = fair
%  private Condition nonFull = monitor.newCondition();
%  private Condition nonEmpty = monitor.newCondition();
%  public void put(T item) throws InterruptedException {
%    monitor.lock();
%    try {
%      while (queue.size() == Capacity) { nonFull.await(); }
%      queue.add(item); nonEmpty.signal();
%    } finally { monitor.unlock(); } }
%  public T get() throws InterruptedException {
%	monitor.lock();
%	try {
%	  while (queue.size() == 0) { nonEmpty.await(); }
%	  T item = queue.remove();
%	  nonFull.signal();
%	  return item;
%	} finally { monitor.unlock(); } } }
%\end{lstlisting}

%\subsubsection{Read-Write Locks - Java}
%
%\begin{lstlisting}
%var rwLock = new ReentrantReadWriteLock(true);
%rwLock.readLock().lock(); // read-only accesses (Shared Lock)
%rwLock.readLock().unlock();
%rwLock.writeLock().lock(); // read and write accesses (Exclusive Lock)
%rwLock.writeLock().unlock();
%\end{lstlisting}

%\subsubsection{CountDownLatch - Java}
%
%\begin{lstlisting}
%public class CarRaceWithLatch {
%  private static final int CARS = 10;
%  private final CountDownLatch ready = new CountDownLatch(CARS); // warte auf N cars
%  private final CountDownLatch start = new CountDownLatch(1);
%
%  private class Car extends Thread {
%  	private final int number;
%  	public Car(int number) { this.number = number; }
%
%  	@Override
%  	public void run() {
%  	  try {
%  	  	ready.countDown(); // Car n: Bin bereit; Zähler um 1 dekrementieren
%  	  	start.await(); // Warte auf Startsignal (dass Count Down 0 ist)
%  	  	System.out.println("Car " + number + " drives ");
%  	  } catch (InterruptedException e) { throw new AssertionError(e); } } }
%
%  private void simulate() throws InterruptedException { // von main() aufgerufen
%  	for (int count = 0; count < CARS; count++) { new Car(count).start(); } // führt run() in Threads aus
%  	ready.await(); // Warte, bis alle bereit sind (dass Count Down 0 ist)
%  	System.out.println("Start race ");
%  	start.countDown(); } } // Gib Startsignal
%\end{lstlisting}

%\subsubsection{Cyclic Barrier - Java}
%
%\begin{lstlisting}
%public class CarRaceWithCyclicBarrier2 {
%  private static final int CARS = 10;
%  private static final int ROUNDS = 10;
%
%  private final CyclicBarrier start = new CyclicBarrier(
%  		CARS + 1, () -> { // Anzahl sich treffender Threads
%  			System.out.println("New round start");
%  		});
%
%  private class Car extends Thread {
%  	private final int number;
%  	public Car(int number) { this.number = number; }
%
%  	@Override
%  	public void run() {
%  		try {
%  			for (int round = 0; round < ROUNDS; round++) {
%  				start.await(); // Autos fahren direkt los, sobald alle da sind
%  				System.out.println("Car " + number + " round " + round); }
%  		} catch (InterruptedException | BrokenBarrierException e) { throw new AssertionError(e); } } }
%
%  private void simulate() throws InterruptedException, BrokenBarrierException { // von main() aufgerufen
%  	for (int count = 0; count < CARS; count++) { new Car(count).start(); }
%  	for (int round = 0; round < ROUNDS; round++) { start.await(); }
%  }
%}
%\end{lstlisting}

%\subsubsection{Deadlocks}
%
%\begin{lstlisting}
%class BankAccount {
%  private int balance;
%  public synchronized void transfer(Acc to, int amount) {
%    balance -= amount;
%    to.deposit(amount); } // implizit geschachtelter Lock
%  public synchronized void deposit (int amount) {
%    balance += amount; } }
%a.transfer(b, 20); /* Thread 1 */ b.transfer(a, 50); // Thread 2
%\end{lstlisting}


