\section{Synchronisation}

\begin{lstlisting}
// A,B Ausgaben können durcheinander sein (zeitlich entkoppelt, Threads laufen beliebig verzahnt/parallel)
public class MultiThreadTest {
    public static void main (String[] args ) {
        // Thread-Implementierung
        var a = new Thread(Thread(() -> multiPrint("A"));
        var b = new Thread(Thread(() -> multiPrint("B"));
        a.start(); b.start(); // richtiger Thread wird hier erzeugt und gestartet (führt run()-Methode des Runnable Interface aus)
        a.join() /* blockiert solange Thread a läuft */ b.join(); }
    static void multiPrint (String label) {
        for (int i = 0; i < 10; i++) {
            System.out.println(label + ": " + i);
            Thread.sleep(100) // versetzt laufenden Thread in Wartezustand
        } } }
\end{lstlisting}

\lstinline{Thread.yield()} laufender Thread gibt Prozessor frei und direkt wieder ready \lstinline{currentThread()} gerade ausführende Thread-Instanz \lstinline{t1.setDaemon(true)} markiert Thread als Daemon

% TODO equired?
\begin{minipage}[t]{0.5\linewidth}
    \begin{lstlisting}
// Explizite Runnable-Implementation
class SimpleLogic implements Runnable {
    @Override
    public void run () {
        // thread behavior
    } }
var myThread = new Thread(new SimpleLogic());
myThread.start();
    \end{lstlisting}
\end{minipage}
\begin{minipage}[t]{0.5\linewidth}
    \begin{lstlisting}
// Sub-Klasse von Thread
class SimpleThread extends Thread {
    @Override
    public void run () { // thread behavior } }
var myThread = new SimpleThread(); myThread.start();
    \end{lstlisting}
\end{minipage}

\subsection{Race Condition}
Mehrere Threads lesen dasselbe Objekt und überschreiben diesen Wert basierend auf der erhaltenen Antwort (Bsp. \lstinline{deposit()} ist Critical Section: zuerst \lstinline{balance} lesen dann \lstinline{+= amount}) Zwischen Lesen und Schreiben kann ein anderer Thread die \lstinline{balance} geändert haben $\rightarrow$ Gegenseitiger Ausschluss (Mutual Exclusion)

\subsection{Synchronisationsmechanismen/-primitiven}

\subsubsection{Synchronized (Monitor-Lock) - Java}

\begin{lstlisting}
// deposit und withdraw sind im gegenseitigem Ausschluss
class BankAccount {
    private int balance = 0;
    public synchronized void deposit (int amount) {
        this.balance += amount; // kritischer Abschnitt
        notifyAll(); } // Wecke alle im Monitor wartenden Threads
    public synchronized boolean withdraw (int amount) {
        while (amount > this.balance) { wait(); } // gibt Lock temporär frei und wartet auf Bedingung (im Monitor Model)
        this.balance -= amount; }
// synchronized(object){ statements } ist eine explizite Angabe, auf welcher Instanz gelockt wird:
public void deposit(int amount) { // explizite Angabe auf welcher Instanz gelockt wird
    synchronized(this) {this.balance += amount;} }
\end{lstlisting}

\lstinline{notify()} signalisiert/weckt nur einen beliebigen (zufälligen), wartenden Thread (Achtung: wartet vielleicht auf andere Bedingung, führt zu ewigem Warten)

\begin{lstlisting}
class BoundedBuffer <T> {
  private Queue<T> queue = new LinkedList<>();
  private int limit = 1; // or initialize in constructor
  public synchronized void put(T item) throws InterruptedException {
    while (queue.size () == limit) { wait(); } // await non full
    queue.add(item); notifyAll(); } // signal non empty
  public synchronized T get() throws InterruptedException {
    while (queue.size () == 0) { wait(); } // await non empty
    var item = queue.remove(); notifyAll(); // non full
    return item; } }
\end{lstlisting}

\subsubsection{Semaphor - Java}

\begin{lstlisting}
// Mutex Semaphore als alternative zu synchronized
class BoundedBuffer <T> {
  private Queue<T> queue = new LinkedList<>();
  private Semaphore upperLimit = new Semaphore(CAPACITY, true); // Initialisierung mit cap freien Ressourcen, true = fair
  private Semaphore lowerLimit = new Semaphore(0, true);
  // private Semaphore mutex = new Semaphore(1, true); // ersetzt synchronized/Monitor-Lock für Mutual exclusion
  public void put(T item) throws InterruptedException {
    upperLimit.acquire(); // bezieht freie Ressource (oder wartet, wenn keine verfügbar)
    synchronized (queue) { queue.add (item); }
    // mutex.acquire(); queue.add(item); mutex.release();
    lowerLimit.release(); } // gibt Ressource frei, benachrichtigt Wartende
  public T get() throws InterruptedException {
    T item;
    lowerLimit.acquire();
    synchronized (queue) { item = queue.remove(); }
    // mutex.acquire(); T item = queue.remove(); mutex.release();
    upperLimit.release();
    return item; } }
\end{lstlisting}

\lstinline{acquire(int permits)} wartet, solange Zähler $<$ permits ist

\subsubsection{Lock \& Condition - Java}
\begin{lstlisting}
class BoundedBuffer<T> {
  private Queue<T> queue = new LinkedList<>();
  private Lock monitor = new ReentrantLock(true); // Lock-Objekt, Sperre für Eintritt in Monitor, true = fair
  private Condition nonFull = monitor.newCondition();
  private Condition nonEmpty = monitor.newCondition();
  public void put(T item) throws InterruptedException {
    monitor.lock();
    try {
      while (queue.size() == Capacity) { nonFull.await(); }
      queue.add(item); nonEmpty.signal();
    } finally { monitor.unlock(); } }
  public T get() throws InterruptedException {
	monitor.lock();
	try {
	  while (queue.size() == 0) { nonEmpty.await(); }
	  T item = queue.remove();
	  nonFull.signal();
	  return item;
	} finally { monitor.unlock(); } } }
\end{lstlisting}

\subsubsection{Read-Write Locks - Java}

\begin{lstlisting}
var rwLock = new ReentrantReadWriteLock(true);
rwLock.readLock().lock(); // read-only accesses (Shared Lock)
rwLock.readLock().unlock();
rwLock.writeLock().lock(); // read and write accesses (Exclusive Lock)
rwLock.writeLock().unlock();
\end{lstlisting}

\subsubsection{CountDownLatch - Java}

\begin{lstlisting}
public class CarRaceWithLatch {
  private static final int CARS = 10;
  private final CountDownLatch ready = new CountDownLatch(CARS); // warte auf N cars
  private final CountDownLatch start = new CountDownLatch(1);

  private class Car extends Thread {
  	private final int number;
  	public Car(int number) { this.number = number; }

  	@Override
  	public void run() {
  	  try {
  	  	ready.countDown(); // Car n: Bin bereit; Zähler um 1 dekrementieren
  	  	start.await(); // Warte auf Startsignal (dass Count Down 0 ist)
  	  	System.out.println("Car " + number + " drives ");
  	  } catch (InterruptedException e) { throw new AssertionError(e); } } }

  private void simulate() throws InterruptedException { // von main() aufgerufen
  	for (int count = 0; count < CARS; count++) { new Car(count).start(); } // führt run() in Threads aus
  	ready.await(); // Warte, bis alle bereit sind (dass Count Down 0 ist)
  	System.out.println("Start race ");
  	start.countDown(); } } // Gib Startsignal
\end{lstlisting}

\subsubsection{Cyclic Barrier - Java}

\begin{lstlisting}
public class CarRaceWithCyclicBarrier2 {
  private static final int CARS = 10;
  private static final int ROUNDS = 10;

  private final CyclicBarrier start = new CyclicBarrier(
  		CARS + 1, () -> { // Anzahl sich treffender Threads
  			System.out.println("New round start");
  		});

  private class Car extends Thread {
  	private final int number;
  	public Car(int number) { this.number = number; }

  	@Override
  	public void run() {
  		try {
  			for (int round = 0; round < ROUNDS; round++) {
  				start.await(); // Autos fahren direkt los, sobald alle da sind
  				System.out.println("Car " + number + " round " + round); }
  		} catch (InterruptedException | BrokenBarrierException e) { throw new AssertionError(e); } } }

  private void simulate() throws InterruptedException, BrokenBarrierException { // von main() aufgerufen
  	for (int count = 0; count < CARS; count++) { new Car(count).start(); }
  	for (int round = 0; round < ROUNDS; round++) { start.await(); }
  }
}
\end{lstlisting}

\subsection{Gefahren}

\subsubsection{Race Conditions}

\subsubsection{Deadlocks}

\begin{lstlisting}
class BankAccount {
  private int balance;
  public synchronized void transfer(Acc to, int amount) {
    balance -= amount;
    to.deposit(amount); } // implizit geschachtelter Lock
  public synchronized void deposit (int amount) {
    balance += amount; } }
a.transfer(b, 20); /* Thread 1 */ b.transfer(a, 50); // Thread 2
\end{lstlisting}


\subsubsection{Starvation}
