\section{Concurrency - JavaScript}
Grundsätzlich Single-Threaded mit einem Event-Loop, \textcolor{red}{schützt nicht vor Race Conditions}

\textbf{Promise} CompletableFuture \lstinline{Promise.all()} \lstinline{Promise.any()} läuft bis zum await synchron

\subsection{Web-Worker} entspricht Thread im Browser, Datenaustausch primär über Messaging (FIFO, Pass by copy), Langlebiger "Prozess": Bleiben nach Start aktiv (mit eigenem Event-Loop), um Messages abzuarbeiten

\begin{lstlisting}
const worker = new Worker('worker.js'); // Quelldatei mit dem Code für den Worker
function computeFibonacci() {
  const input = 42;
  setStatusText(`computing fibonacci(${input})`);
  worker.postMessage(input); } // Message an Worker senden

worker.onmessage = event => { // Callback für die eingehenden Messages registrieren
  const { input, output } = event.data; // Inhalt der Message
  setStatusText(`fibonacci(${input})=${output}`);
};

function setStatusText(text) { document.getElementById('status').innerText = text; }

/* worker.js */
onmessage = message => {
  const n = message.data;
  postMessage({ input: n, output: fibonacci(n) }); }; //	Antwort an den Thread welcher den Worker gestartet hat

function fibonacci(n) {
  if(n === 0) { return 0; }
  if(n === 1) { return 1; }
  return fibonacci(n - 1) + fibonacci(n - 2); }
\end{lstlisting}

\columnbreak

\section{Concurrency - Python}

\textbf{Global Interpreter Lock (GIL)} Nur ein Thread kann Python Byte-Code ausführen, kein Speedup für CPU-Bound Operationen möglich (Ausnahme: Libraries geben GIL frei, Bsp. Numpy), Data Races dennoch möglich (Reordering möglich, Visibility nicht garantiert) $\rightarrow$ kein Memory Model definiert

\textbf{Shared Memory} muss explizit definiert werden $\rightarrow$ \lstinline{multiprocessing.Value}, erfordert Angabe eines Type-Codes oder ctypes

\begin{lstlisting}[style=Python]
from multiprocessing import Process
...
result = Value('i', -1, lock=False)  oder: Value(c_int, -1 ''' Initialwert des Shared-Memory ''', lock=False)
process = Process(target=compute_fibonacci, args=(10, result))
...
def compute_fibonacci(n, result ) # Hilfsfunktion zum Speichern des Resultats
  result.value = fibonacci(n)
\end{lstlisting}


\subsection{Threads}

\begin{lstlisting}[style=Python]
from threading import Thread
if __name__ == '__main__':
  t = Thread(target=fibonacci, args=(10,) )
  t.start()
  ...
  t.join()
\end{lstlisting}

\subsection{Prozesse}

\begin{minipage}[t]{0.5\linewidth}
    \begin{lstlisting}[style=Python]
from multiprocessing import Process
# bei Process zwingend notwendig (sonst wird der start immer wieder neu aufgerufen)!
if __name__ == '__main__':
  p = Process(target=fibonacci, args=(10,) ) # Funktionen und Argumente müssen "picklebar" (serialisierbar) sein
  p.start()
  ...
  p.join()
    \end{lstlisting}
\end{minipage}
\begin{minipage}[t]{0.48\linewidth}
    \begin{lstlisting}[style=Python]
from multiprocessing import Process
if __name__ == '__main__':
  total_primes = 0
  results = [Value('i', 0, lock=False) for _ in range(TOTAL_PROCESSES)]
  processes = [Process(target=is_prime_process, args=(i, results[i])) for i in range(TOTAL_PROCESSES)]
  for process in processes:
    process.start()
  for i, process in enumerate(processes):
    process.join()
    total_primes += results[i].value
    \end{lstlisting}
\end{minipage}


\subsection{Pools}

\begin{lstlisting}[style=Python]
from concurrent.futures.process import ProcessPoolExecutor
# oder from concurrent.futures.thread import ThreadPoolExecutor

if __name__ == '__main__':
  with ProcessPoolExecutor() as pool:
  	future = pool.submit(fibonacci, 10)
  	print(future.result()) # Blockiert, bis Task beendet wurde

    # gleich zu berechnende Liste übergeben
  	multiple = pool.map(fibonacci, [1, 2, 4, 10, 30])
  	for result in multiple:
  	  print(result) # Blockierender Generator: Wartet auf jeweiliges Resultat (Aufrufreihenfolge wird berücksichtigt)
    all_results = list(multiple) # Alternative
\end{lstlisting}

\subsection{asyncio (async/await)}

\textcolor{red}{Coroutinen werden erst beim zugehörigen await ausgeführt} keine parallele Ausführung, ausser Coroutine wird als Task verpackt $\rightarrow$ \lstinline{asyncio.create_task(<Coroutine>)} / \lstinline{asyncio.gather(<Coroutine1>, <Coroutine2>, ...)})

\begin{lstlisting}[style=Python]
import asyncio
async def sub_routine(n): # Coroutine-Functions
  print(f'Start {n}')
  await asyncio.sleep(1)
  print(f'Ende {n}')

async def main():
  coroutine1 = sub_routine(1) # return: Coroutine-Object
  coroutine2 = sub_routine(2)
  await coroutine2
  await coroutine1

if __name__ == '__main__':
  asyncio.run(main()) # start-coroutine, startet asyncio Event-Loop
\end{lstlisting}
Keine parallele Ausführung. → async
Methoden werden erst beim zugehörigem await ausgeführt


