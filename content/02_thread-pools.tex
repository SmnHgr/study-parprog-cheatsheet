\section{Thread Pools}

Worker-Threads sind Daemon Threads

\textcolor{blue}{Run to completion} jeder Thread hat Stack, bleibt bei verschiedenen Tasks derselbe

\textcolor{blue}{Fire and Forget} wird nicht bis Thread-Ende gewartet

\textcolor{blue}{Caller-zentrisch} Aufrufende Methode wartet auf Beendigung der zurückgegebenen Futures

\textcolor{blue}{Callee-zentrisch} Jede asynchrone Methode löst nach seiner Beendigung selbstständig eine Continuation aus

\subsection{Fork-Join-Pool - Java}

\textbf{Rekursive Tasks}
Tasks können Untertasks starten und abwarten, abhängig von Threshold/Array-Länge

\begin{lstlisting}
class CountTask extends RecursiveTask<Integer> { // RecursiveAction = void
  private final static int THRESHOLD = 1;
  private final int from, to; private final int[] array;

  public CountTask(int from, int to) { this.from=from; this.to=to; }

  // Task-Implementierung
  protected Integer compute() {
    // Tuning mit Schwellenwert (keine Über-Parallelisierung
    if(to - from > THRESHOLD) {
      int middle = (from + to) / 2;
\end{lstlisting}
\begin{minipage}[t]{0.6\linewidth}
    \begin{lstlisting}
      // parallel count (task)
      var left = new CountTask (from, middle);
      var right = new CountTask (middle, to);
      // als Sub-Task in anderem Task starten
      left.fork(); right.fork();
      // auf Task-Ende warten / Resultat abfragen
      return right.join () + left.join(); }
    \end{lstlisting}
\end{minipage}
\begin{minipage}[t]{0.39\linewidth}
    \begin{lstlisting}
// parallel action (void return)
invokeAll(
  new UpdateAr(array, fro, mid),
  new UpdateAr(array, mid, to) );
    \end{lstlisting}
\end{minipage}

\begin{lstlisting}
    else {
      // sequential count
      int count = 0;
      for (int num = from; num < to; num++){ if (isPrime(num)) { count++; }}
      return count; } } }
// Start
// blockiert, startet ein Sub-Task und wartet ab
int result = new CountTask(2, INPUT_SIZE).invoke(); // Standard Pool
var threadPool = new ForkJoinPool(); // moderner Thread Pool
int result = threadPool.invoke(new CountTask(2, INPUT_SIZE));//expliz. ThreadPool

// Future Verwendung: blockiert nicht, lanciert Task ohne Warten
Future<Integer> future /* hält zukünftiges Resultat */ = threadPool.submit(
    () -> { /* longOperation */ return value; }); // Rückgabe möglich
int result = future.get() // Blockiert, bis Task beendet (Fehler=ExecutionExcep.)
\end{lstlisting}

\lstinline{invokeAll(a, b)} equals \lstinline{a.fork(); b.fork(); b.join(); a.join(); } Mehrere Sub-Tasks starten und abwarten

\textbf{Asynchrone Programmierung}
Task verketten, hängt Folgeaufgabe an asynchrone Aufgabe
\begin{lstlisting}
CompletableFuture<int> future = CompletableFuture
.supplyAsync(()->longOperation()).thenApply(second); // Continuation mit return
.runAsync().thenAccept(result -> System.out.println("Continuation")) // kein return (Fire and Forget)

// Multi-Continuation
CompletableFuture.allOf(fut1, fu2) // Warten aufeinander
CompletableFuture.any(fu1, fu2) // Sobald einer fertig ist
\end{lstlisting}




\subsection{Task Parallel Library (TPL) - .NET (\#C)}
läuft auf Anzahl Worker Threads (Anzahl CPUs)

\begin{lstlisting}
var myThread = new Thread(() => {
  for (int i = 0; i < 100; i ++) {
    Console.WriteLine("MyThread step {0}", i); } });
myThread.Start(); myThread.Join();
\end{lstlisting}

%\begin{lstlisting}
%// Monitor in .NET
%class BankAccount {
%  private decimal balance;
%  private object syncObject = new() // Monitor Hilfsobjekt
%  public void Withdraw(decimal amount) {
%    lock (syncObject) { // Analog zu synchronized Stmt
%      while (amount > balance) { // Schlaufe auch notwendig
%        Monitor.Wait(syncObject); } // kein Spurious Wakeup
%      balance -= amount; } }
%  public void Deposit(decimal amount) {
%    lock (syncObject) {
%      balance += amount;
%      Monitor.PulseAll(syncObject) }}} // wie notifyAll
%\end{lstlisting}

\textbf{Task Parallelität - Explizite Tasks starten und warten}
\begin{lstlisting}
Task<int> /* Rückgabetyp */ task = Task.Run(() => {
  var left = Task.Run(() => Count(leftPart)); // starte Sub-Task
  var right = Task.Run(() => Count(rightPart)); // starte Sub-Task
  int total = left.Result + right.Result; // warte auf Sub-Tasks
  return total; }); // task implementation

// perform other activity
task.Wait(); // Blockiert, bis Task beendet ist
Console.Write(task.Result); // Blockiert bis Task Ende und liefert dann Resultat
\end{lstlisting}

\textbf{Datenparallelität - Parallele Statements und Queries }
\begin{lstlisting}
// Barriere der Tasks jeweils am Ende
Parallel.Invoke( () => MergeSort(l, m), () => MergeSort(m, r) ); // Statements
// Iteration unabhängig/parallel ausführen, Barriere der Tasks am Ende
Parallel.ForEach(list, file => Convert(file)); // Loop, Schlaufen-Bodies
Parallel.For(0, array.Length, i => DoComputation(array[i])); // Iterator
\end{lstlisting}

\textcolor{blue}{Parallele Loop Partitionierung}

Schlaufe mit vielen sehr kurzen Bodies ist ineffizient (Ausführung jeder Iteration als parallelen Tasks) \textcolor{darkGreen}{TPL gruppiert/partitioniert autom. mehrere Bodies zu Task entsprechend der verfügbaren Worker Threads} Range Partitioner bei \lstinline{Parallel.For}, \lstinline{Parallel.Invoke}

%\textbf{Parallel LINQ (PLINQ)}
%
%\begin{lstlisting}
%// Beliebige Reihenfolge als Resultat
%from book in bookCollection.AsParallel()
%  where book.Title.Contains("Concurrency")
%  select book.ISBN
%
%// Reihenfolge erhalten (Performance-Nachteil)
%from number in inputList.AsParallel().AsOrdered()
%  select IsPrime(number)
%\end{lstlisting}

\textbf{Asynchrone Programmierung - Mit Continuation Style}

\begin{lstlisting}
var task = Task.Run(LongOperation); // Task repräsentiert asynchronen Aufruf
// perform other work
int result = task.Result; // Zugriff ähnlich wie mit Future
// Task Continuations (Wie CompletableFutures)
task1.ContinueWith(task2).ContinueWith(task3);
// Multi-Continuation
Task.WhenAll(task1, task2).ContinueWith(continuation);
Task.WhenAny(task1, task2).ContinueWith(continuation);
\end{lstlisting}
